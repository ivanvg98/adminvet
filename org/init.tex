% Created 2019-06-10 lun 10:29
% Intended LaTeX compiler: pdflatex
\documentclass[11pt]{article}
\usepackage[utf8]{inputenc}
\usepackage[T1]{fontenc}
\usepackage{graphicx}
\usepackage{grffile}
\usepackage{longtable}
\usepackage{wrapfig}
\usepackage{rotating}
\usepackage[normalem]{ulem}
\usepackage{amsmath}
\usepackage{textcomp}
\usepackage{amssymb}
\usepackage{capt-of}
\usepackage{hyperref}
\author{Vladimir Azpeitia Hernández}
\date{\today}
\title{Proyecto para la administración de la Veterinaria "Pony"}
\hypersetup{
 pdfauthor={Vladimir Azpeitia Hernández},
 pdftitle={Proyecto para la administración de la Veterinaria "Pony"},
 pdfkeywords={},
 pdfsubject={},
 pdfcreator={Emacs 25.2.2 (Org mode 9.2.3)}, 
 pdflang={English}}
\begin{document}

\maketitle
\section{Planeacion}
\label{sec:orgf28371c}
\subsection{Recoleccion de documentos}
\label{sec:orgb1537d7}
\subsection{Actvidades principales}
\label{sec:org1c1d314}
\begin{enumerate}
\item Control de consultas
Hay un local en donde las consultas llegan de dos formas: una através del telefono de casa y otra es cuando los
clientes llegan personalmente.

Otra manera de obtener consultas es en el viaje
\item Control de clientes(créditos)
\item Control de proveedores
\item Cartilla perros
\end{enumerate}
\section{Diseño}
\label{sec:org6a66c79}
\subsection{Diseño de la base de datos}
\label{sec:orgbf20072}
\textbf{Posibles entidades:} \\
\begin{center}
\begin{tabular}{lll}
Entidad & Atributos & Observaciones\\
\hline
Clientes &  & \\
Veterinarios &  & \\
Consultas &  & \\
Tratamiento &  & \\
Medicamentos &  & \\
 &  & \\
\end{tabular}
\end{center}

\section{Implementación}
\label{sec:orgeaa4d4f}
\section{Pruebas / Mantenimiento y refinamiento}
\label{sec:orgd3b6180}
\end{document}
